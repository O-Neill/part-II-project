%%%%%%%%%%%%%%% Updated by MR March 2007 %%%%%%%%%%%%%%%%
\documentclass[12pt]{article}
\usepackage{a4wide}

\newcommand{\al}{$<$}
\newcommand{\ar}{$>$}

\parindent 0pt
\parskip 6pt

\begin{document}

\thispagestyle{empty}
\rightline{\large \textbf{Benjamin O'Neill}}

\vfil

%cover page
\centerline{\Large\bf Efficient Asymmetric Cryptography for}
\centerline{\Large\bf RFID Access Control}
\vspace{0.4in}
\centerline{\Large Computer Science Tripos - Part II}
\vspace{0.3in}
\centerline{\Large Emmanuel College}
\vspace{0.3in}
\centerline{\Large 2018}
\vfil
\pagebreak

%Proforma
\thispagestyle{empty}
\centerline{\Large Proforma}

\noindent
\begin{minipage}{0.3\textwidth}
\raggedright
{\bf Name:}\\
{\bf College:}\\
{\bf Project Title:}\\
{\bf Examination:}\\
{\bf Word count:}\\
{\bf Project Originator:}\\
{\bf Project Supervisor:}
\end{minipage}
\begin{minipage}{0.7\textwidth}
%\raggedleft
Benjamin O'Neill\\
Emmanuel College\\
Efficient Asymmetric Cryptography for RFID Access Control\\
Computer Science Tripos Part II - 2018\\
TODO\\
Dr Markus Kuhn\\
Dr Markus Kuhn
\end{minipage}

\vspace{0.3in}
{\large\bf Original aims of the project}\\
TODO: Summarise project proposal.

\vspace{0.3in}
{\large\bf Work completed}\\
TODO:

\vspace{0.3in}
{\large\bf Special difficulties faced}\\
TODO: Maybe mention library incompatibilities, broken cards.
\pagebreak

%Declaration of originality page
\centerline{\Large Declaration of Originality}
I, Benjamin O'Neill of Emmanuel College, being a candidate for Part II of the Computer Science Tripos, hereby declare that this dissertation and the work described in it are my own work, unaided except as may be specified below, and that the dissertation does not contain material that has already been used to any substantial extent for a comparable purpose.
% TODO: Is this an issue given that a previous student did the same dissertation?

Signed, Benjamin O'Neill

\vspace{0.2in}
Date: TODO
\pagebreak

%Table of contents
\section*{Table of Contents}
TODO\\
Look into how to link to things using \LaTeX.
\pagebreak

%introduction
\section{Introduction}
TODO\\
Motivation for the project. What needs are being addressed.\\
Crypto description, symmetric and symmetric.\\
Background information about smartcards.\\
current university smartcard system, shortcomings. (Look into MIFARE protocol).\\
Detail security flaws in MIFARE classic. Uni may be moving to MIFARE plus.\\
Information about JavaCard and development process.
\pagebreak

%Preparation
\section{Preparation}
TODO\\
Work undertaken before code was written. How proposal was refined and clarified.\\
Analysis of problems with MIFARE classic. Outline main problems.\\
Quick analysis of different protocols and comparisons.\\
Reading about smartcard and java card stuff. ISO7816 standards to get background understanding, descriptions of java card on oracle website.\\
APDU stuff\\
Research into different smartcards, readers etc.\\
Researched asymmetric cryptography, and when OPACITY chosen, researched elliptic curve cryptography. \\
"Requirements analysis" section where project proposal points elaborated. Reference other software engineering techniques.\\
Familiarised with various misc things like ASN1, had to learn various quirks and limitations of restricted Java Card language (e.g. int-to-short casting, restrictions on APDU passing, different garbage collection system). Very little online help about any of this.\\
Analysis of available tools, with reference to different JC versions (eclipse plugin for 3.0.5 most common, ant/command line more common in older ones. Used both ANT and command line when appropriate with 3.0.4)\\
Also GPShell, specific smartcard details, versions purchased.\\

\pagebreak

%Implementation
\section{Implementation}
TODO\\
What was actually produced. Programs/code written.\\
Could move a few things to here depending on whether learning the dev process counts as preparation or implementation.\\
Give examples of useful code abstractions.\\
Details of design decisions, libraries used, code produced (i.e. non-JCMathLib Java Card code including CMAC and ECDSA. (TODO: implement ECDSA)\\
Use small code fragments to illustrate certain things, but not too much here.\\
Properly point out JCMathLib and other 3rd party things. Point out that license is compatible with the context I am using it in.\\
Can highlight major milestones in the process.\\
How is card issued? Prob issuer gets Id, adds it to allowed IDs of all terminals. TODO: Is this right?\\
XML mock PB store.
\pagebreak

%Evaluation
\section{Evaluation}
TODO\\
Assessors look for signs of success, thorough and systematic evaluation. (see pink book 8.3)\\
Give sample output, timing, perhaps photographs of current system with discussion of how it would be implemented in reality.\\
APDU trace\\
Runtime for initial encounter vs later counter using PB\\
Analysis of how much time each operation actually takes.\\
Were original goals achieved?\\
Do a brief security analysis of the system, considering security of the protocol itself, and other things e.g. whether issuing process has faults, whether PB entries can be altered maliciously etc.\\
Some residual bugs expected. If any exist, explain briefly. Show it still works in the basic case.\\
What general shortcomings does it have?
\pagebreak

%Conclusions
\section{Conclusions}
TODO\\
Don't introduce new information here, just summarise and briefly discuss.\\
What I have learned generally about the area, and what should I do differently in future when doing projects.\\
Should be short, refer to introduction. How would project have been planned or executed differently if done again.\\
What more could be done.
\pagebreak

%Bibliography
\section{Bibliography}
TODO
\pagebreak

%Appendices
\section{Appendices}
TODO\\
Sample code, protocol diagram and other details, other details about the workings of smartcards e.g. from ISO7816, Java Card documentation.
\pagebreak

%Index
\section{Index}
TODO (Optional)
\pagebreak

\end{document}
